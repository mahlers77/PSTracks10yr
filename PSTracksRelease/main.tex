\documentclass[aps,10pt,prd,twocolumn,floats,letterpaper,showpacs,nofootinbib,bibnotes,notitlepage,superscriptaddress]{revtex4-1}

\usepackage{graphicx}
\usepackage{mathrsfs}
\usepackage[intlimits,centertags]{amsmath}
\usepackage{amssymb,amsfonts}
\usepackage[pdftex]{hyperref}
\usepackage[x11names]{xcolor}
\usepackage{enumerate}
\usepackage{xfrac}
\usepackage[normalem]{ulem}

%\linenumbers

\begin{document}

\title{IceCube Data for Neutrino Point-Source Searches: Years 2008--2018}

\affiliation{III. Physikalisches Institut, RWTH Aachen University, D-52056 Aachen, Germany}
\affiliation{Department of Physics, University of Adelaide, Adelaide, 5005, Australia}
\affiliation{Dept. of Physics and Astronomy, University of Alaska Anchorage, 3211 Providence Dr., Anchorage, AK 99508, USA}
\affiliation{Dept. of Physics, University of Texas at Arlington, 502 Yates St., Science Hall Rm 108, Box 19059, Arlington, TX 76019, USA}
\affiliation{CTSPS, Clark-Atlanta University, Atlanta, GA 30314, USA}
\affiliation{School of Physics and Center for Relativistic Astrophysics, Georgia Institute of Technology, Atlanta, GA 30332, USA}
\affiliation{Dept. of Physics, Southern University, Baton Rouge, LA 70813, USA}
\affiliation{Dept. of Physics, University of California, Berkeley, CA 94720, USA}
\affiliation{Lawrence Berkeley National Laboratory, Berkeley, CA 94720, USA}
\affiliation{Institut f{\"u}r Physik, Humboldt-Universit{\"a}t zu Berlin, D-12489 Berlin, Germany}
\affiliation{Fakult{\"a}t f{\"u}r Physik {\&} Astronomie, Ruhr-Universit{\"a}t Bochum, D-44780 Bochum, Germany}
\affiliation{Universit{\'e} Libre de Bruxelles, Science Faculty CP230, B-1050 Brussels, Belgium}
\affiliation{Vrije Universiteit Brussel (VUB), Dienst ELEM, B-1050 Brussels, Belgium}
\affiliation{Dept. of Physics, Massachusetts Institute of Technology, Cambridge, MA 02139, USA}
\affiliation{Dept. of Physics and Institute for Global Prominent Research, Chiba University, Chiba 263-8522, Japan}
\affiliation{Dept. of Physics and Astronomy, University of Canterbury, Private Bag 4800, Christchurch, New Zealand}
\affiliation{Dept. of Physics, University of Maryland, College Park, MD 20742, USA}
\affiliation{Dept. of Astronomy, Ohio State University, Columbus, OH 43210, USA}
\affiliation{Dept. of Physics and Center for Cosmology and Astro-Particle Physics, Ohio State University, Columbus, OH 43210, USA}
\affiliation{Niels Bohr Institute, University of Copenhagen, DK-2100 Copenhagen, Denmark}
\affiliation{Dept. of Physics, TU Dortmund University, D-44221 Dortmund, Germany}
\affiliation{Dept. of Physics and Astronomy, Michigan State University, East Lansing, MI 48824, USA}
\affiliation{Dept. of Physics, University of Alberta, Edmonton, Alberta, Canada T6G 2E1}
\affiliation{Erlangen Centre for Astroparticle Physics, Friedrich-Alexander-Universit{\"a}t Erlangen-N{\"u}rnberg, D-91058 Erlangen, Germany}
\affiliation{Physik-department, Technische Universit{\"a}t M{\"u}nchen, D-85748 Garching, Germany}
\affiliation{D{\'e}partement de physique nucl{\'e}aire et corpusculaire, Universit{\'e} de Gen{\`e}ve, CH-1211 Gen{\`e}ve, Switzerland}
\affiliation{Dept. of Physics and Astronomy, University of Gent, B-9000 Gent, Belgium}
\affiliation{Dept. of Physics and Astronomy, University of California, Irvine, CA 92697, USA}
\affiliation{Karlsruhe Institute of Technology, Institut f{\"u}r Kernphysik, D-76021 Karlsruhe, Germany}
\affiliation{Dept. of Physics and Astronomy, University of Kansas, Lawrence, KS 66045, USA}
\affiliation{SNOLAB, 1039 Regional Road 24, Creighton Mine 9, Lively, ON, Canada P3Y 1N2}
\affiliation{Department of Physics and Astronomy, UCLA, Los Angeles, CA 90095, USA}
\affiliation{Department of Physics, Mercer University, Macon, GA 31207-0001, USA}
\affiliation{Dept. of Astronomy, University of Wisconsin, Madison, WI 53706, USA}
\affiliation{Dept. of Physics and Wisconsin IceCube Particle Astrophysics Center, University of Wisconsin, Madison, WI 53706, USA}
\affiliation{Institute of Physics, University of Mainz, Staudinger Weg 7, D-55099 Mainz, Germany}
\affiliation{Department of Physics, Marquette University, Milwaukee, WI, 53201, USA}
\affiliation{Institut f{\"u}r Kernphysik, Westf{\"a}lische Wilhelms-Universit{\"a}t M{\"u}nster, D-48149 M{\"u}nster, Germany}
\affiliation{Bartol Research Institute and Dept. of Physics and Astronomy, University of Delaware, Newark, DE 19716, USA}
\affiliation{Dept. of Physics, Yale University, New Haven, CT 06520, USA}
\affiliation{Dept. of Physics, University of Oxford, Parks Road, Oxford OX1 3PU, UK}
\affiliation{Dept. of Physics, Drexel University, 3141 Chestnut Street, Philadelphia, PA 19104, USA}
\affiliation{Physics Department, South Dakota School of Mines and Technology, Rapid City, SD 57701, USA}
\affiliation{Dept. of Physics, University of Wisconsin, River Falls, WI 54022, USA}
\affiliation{Dept. of Physics and Astronomy, University of Rochester, Rochester, NY 14627, USA}
\affiliation{Oskar Klein Centre and Dept. of Physics, Stockholm University, SE-10691 Stockholm, Sweden}
\affiliation{Dept. of Physics and Astronomy, Stony Brook University, Stony Brook, NY 11794-3800, USA}
\affiliation{Dept. of Physics, Sungkyunkwan University, Suwon 16419, Korea}
\affiliation{Institute of Basic Science, Sungkyunkwan University, Suwon 16419, Korea}
\affiliation{Dept. of Physics and Astronomy, University of Alabama, Tuscaloosa, AL 35487, USA}
\affiliation{Dept. of Astronomy and Astrophysics, Pennsylvania State University, University Park, PA 16802, USA}
\affiliation{Dept. of Physics, Pennsylvania State University, University Park, PA 16802, USA}
\affiliation{Dept. of Physics and Astronomy, Uppsala University, Box 516, S-75120 Uppsala, Sweden}
\affiliation{Dept. of Physics, University of Wuppertal, D-42119 Wuppertal, Germany}
\affiliation{DESY, D-15738 Zeuthen, Germany}

\author{M. G. Aartsen}
\affiliation{Dept. of Physics and Astronomy, University of Canterbury, Private Bag 4800, Christchurch, New Zealand}
\author{M. Ackermann}
\affiliation{DESY, D-15738 Zeuthen, Germany}
\author{J. Adams}
\affiliation{Dept. of Physics and Astronomy, University of Canterbury, Private Bag 4800, Christchurch, New Zealand}
\author{J. A. Aguilar}
\affiliation{Universit{\'e} Libre de Bruxelles, Science Faculty CP230, B-1050 Brussels, Belgium}
\author{M. Ahlers}
\affiliation{Niels Bohr Institute, University of Copenhagen, DK-2100 Copenhagen, Denmark}
\author{M. Ahrens}
\affiliation{Oskar Klein Centre and Dept. of Physics, Stockholm University, SE-10691 Stockholm, Sweden}
\author{C. Alispach}
\affiliation{D{\'e}partement de physique nucl{\'e}aire et corpusculaire, Universit{\'e} de Gen{\`e}ve, CH-1211 Gen{\`e}ve, Switzerland}
\author{K. Andeen}
\affiliation{Department of Physics, Marquette University, Milwaukee, WI, 53201, USA}
\author{T. Anderson}
\affiliation{Dept. of Physics, Pennsylvania State University, University Park, PA 16802, USA}
\author{I. Ansseau}
\affiliation{Universit{\'e} Libre de Bruxelles, Science Faculty CP230, B-1050 Brussels, Belgium}
\author{G. Anton}
\affiliation{Erlangen Centre for Astroparticle Physics, Friedrich-Alexander-Universit{\"a}t Erlangen-N{\"u}rnberg, D-91058 Erlangen, Germany}
\author{C. Arg{\"u}elles}
\affiliation{Dept. of Physics, Massachusetts Institute of Technology, Cambridge, MA 02139, USA}
\author{J. Auffenberg}
\affiliation{III. Physikalisches Institut, RWTH Aachen University, D-52056 Aachen, Germany}
\author{S. Axani}
\affiliation{Dept. of Physics, Massachusetts Institute of Technology, Cambridge, MA 02139, USA}
\author{P. Backes}
\affiliation{III. Physikalisches Institut, RWTH Aachen University, D-52056 Aachen, Germany}
\author{H. Bagherpour}
\affiliation{Dept. of Physics and Astronomy, University of Canterbury, Private Bag 4800, Christchurch, New Zealand}
\author{X. Bai}
\affiliation{Physics Department, South Dakota School of Mines and Technology, Rapid City, SD 57701, USA}
\author{A. Balagopal V.}
\affiliation{Karlsruhe Institute of Technology, Institut f{\"u}r Kernphysik, D-76021 Karlsruhe, Germany}
\author{A. Barbano}
\affiliation{D{\'e}partement de physique nucl{\'e}aire et corpusculaire, Universit{\'e} de Gen{\`e}ve, CH-1211 Gen{\`e}ve, Switzerland}
\author{S. W. Barwick}
\affiliation{Dept. of Physics and Astronomy, University of California, Irvine, CA 92697, USA}
\author{B. Bastian}
\affiliation{DESY, D-15738 Zeuthen, Germany}
\author{V. Baum}
\affiliation{Institute of Physics, University of Mainz, Staudinger Weg 7, D-55099 Mainz, Germany}
\author{S. Baur}
\affiliation{Universit{\'e} Libre de Bruxelles, Science Faculty CP230, B-1050 Brussels, Belgium}
\author{R. Bay}
\affiliation{Dept. of Physics, University of California, Berkeley, CA 94720, USA}
\author{J. J. Beatty}
\affiliation{Dept. of Astronomy, Ohio State University, Columbus, OH 43210, USA}
\affiliation{Dept. of Physics and Center for Cosmology and Astro-Particle Physics, Ohio State University, Columbus, OH 43210, USA}
\author{K.-H. Becker}
\affiliation{Dept. of Physics, University of Wuppertal, D-42119 Wuppertal, Germany}
\author{J. Becker Tjus}
\affiliation{Fakult{\"a}t f{\"u}r Physik {\&} Astronomie, Ruhr-Universit{\"a}t Bochum, D-44780 Bochum, Germany}
\author{S. BenZvi}
\affiliation{Dept. of Physics and Astronomy, University of Rochester, Rochester, NY 14627, USA}
\author{D. Berley}
\affiliation{Dept. of Physics, University of Maryland, College Park, MD 20742, USA}
\author{E. Bernardini}
\affiliation{DESY, D-15738 Zeuthen, Germany}
\thanks{also at Universit{\`a} di Padova, I-35131 Padova, Italy}
\author{D. Z. Besson}
\affiliation{Dept. of Physics and Astronomy, University of Kansas, Lawrence, KS 66045, USA}
\thanks{also at National Research Nuclear University, Moscow Engineering Physics Institute (MEPhI), Moscow 115409, Russia}
\author{G. Binder}
\affiliation{Dept. of Physics, University of California, Berkeley, CA 94720, USA}
\affiliation{Lawrence Berkeley National Laboratory, Berkeley, CA 94720, USA}
\author{D. Bindig}
\affiliation{Dept. of Physics, University of Wuppertal, D-42119 Wuppertal, Germany}
\author{E. Blaufuss}
\affiliation{Dept. of Physics, University of Maryland, College Park, MD 20742, USA}
\author{S. Blot}
\affiliation{DESY, D-15738 Zeuthen, Germany}
\author{C. Bohm}
\affiliation{Oskar Klein Centre and Dept. of Physics, Stockholm University, SE-10691 Stockholm, Sweden}
\author{M. B{\"o}rner}
\affiliation{Dept. of Physics, TU Dortmund University, D-44221 Dortmund, Germany}
\author{S. B{\"o}ser}
\affiliation{Institute of Physics, University of Mainz, Staudinger Weg 7, D-55099 Mainz, Germany}
\author{O. Botner}
\affiliation{Dept. of Physics and Astronomy, Uppsala University, Box 516, S-75120 Uppsala, Sweden}
\author{J. B{\"o}ttcher}
\affiliation{III. Physikalisches Institut, RWTH Aachen University, D-52056 Aachen, Germany}
\author{E. Bourbeau}
\affiliation{Niels Bohr Institute, University of Copenhagen, DK-2100 Copenhagen, Denmark}
\author{J. Bourbeau}
\affiliation{Dept. of Physics and Wisconsin IceCube Particle Astrophysics Center, University of Wisconsin, Madison, WI 53706, USA}
\author{F. Bradascio}
\affiliation{DESY, D-15738 Zeuthen, Germany}
\author{J. Braun}
\affiliation{Dept. of Physics and Wisconsin IceCube Particle Astrophysics Center, University of Wisconsin, Madison, WI 53706, USA}
\author{S. Bron}
\affiliation{D{\'e}partement de physique nucl{\'e}aire et corpusculaire, Universit{\'e} de Gen{\`e}ve, CH-1211 Gen{\`e}ve, Switzerland}
\author{J. Brostean-Kaiser}
\affiliation{DESY, D-15738 Zeuthen, Germany}
\author{A. Burgman}
\affiliation{Dept. of Physics and Astronomy, Uppsala University, Box 516, S-75120 Uppsala, Sweden}
\author{J. Buscher}
\affiliation{III. Physikalisches Institut, RWTH Aachen University, D-52056 Aachen, Germany}
\author{R. S. Busse}
\affiliation{Institut f{\"u}r Kernphysik, Westf{\"a}lische Wilhelms-Universit{\"a}t M{\"u}nster, D-48149 M{\"u}nster, Germany}
\author{T. Carver}
\affiliation{D{\'e}partement de physique nucl{\'e}aire et corpusculaire, Universit{\'e} de Gen{\`e}ve, CH-1211 Gen{\`e}ve, Switzerland}
\author{C. Chen}
\affiliation{School of Physics and Center for Relativistic Astrophysics, Georgia Institute of Technology, Atlanta, GA 30332, USA}
\author{E. Cheung}
\affiliation{Dept. of Physics, University of Maryland, College Park, MD 20742, USA}
\author{D. Chirkin}
\affiliation{Dept. of Physics and Wisconsin IceCube Particle Astrophysics Center, University of Wisconsin, Madison, WI 53706, USA}
\author{S. Choi}
\affiliation{Dept. of Physics, Sungkyunkwan University, Suwon 16419, Korea}
\author{K. Clark}
\affiliation{SNOLAB, 1039 Regional Road 24, Creighton Mine 9, Lively, ON, Canada P3Y 1N2}
\author{L. Classen}
\affiliation{Institut f{\"u}r Kernphysik, Westf{\"a}lische Wilhelms-Universit{\"a}t M{\"u}nster, D-48149 M{\"u}nster, Germany}
\author{A. Coleman}
\affiliation{Bartol Research Institute and Dept. of Physics and Astronomy, University of Delaware, Newark, DE 19716, USA}
\author{G. H. Collin}
\affiliation{Dept. of Physics, Massachusetts Institute of Technology, Cambridge, MA 02139, USA}
\author{J. M. Conrad}
\affiliation{Dept. of Physics, Massachusetts Institute of Technology, Cambridge, MA 02139, USA}
\author{P. Coppin}
\affiliation{Vrije Universiteit Brussel (VUB), Dienst ELEM, B-1050 Brussels, Belgium}
\author{P. Correa}
\affiliation{Vrije Universiteit Brussel (VUB), Dienst ELEM, B-1050 Brussels, Belgium}
\author{D. F. Cowen}
\affiliation{Dept. of Astronomy and Astrophysics, Pennsylvania State University, University Park, PA 16802, USA}
\affiliation{Dept. of Physics, Pennsylvania State University, University Park, PA 16802, USA}
\author{R. Cross}
\affiliation{Dept. of Physics and Astronomy, University of Rochester, Rochester, NY 14627, USA}
\author{P. Dave}
\affiliation{School of Physics and Center for Relativistic Astrophysics, Georgia Institute of Technology, Atlanta, GA 30332, USA}
\author{C. De Clercq}
\affiliation{Vrije Universiteit Brussel (VUB), Dienst ELEM, B-1050 Brussels, Belgium}
\author{J. J. DeLaunay}
\affiliation{Dept. of Physics, Pennsylvania State University, University Park, PA 16802, USA}
\author{H. Dembinski}
\affiliation{Bartol Research Institute and Dept. of Physics and Astronomy, University of Delaware, Newark, DE 19716, USA}
\author{K. Deoskar}
\affiliation{Oskar Klein Centre and Dept. of Physics, Stockholm University, SE-10691 Stockholm, Sweden}
\author{S. De Ridder}
\affiliation{Dept. of Physics and Astronomy, University of Gent, B-9000 Gent, Belgium}
\author{P. Desiati}
\affiliation{Dept. of Physics and Wisconsin IceCube Particle Astrophysics Center, University of Wisconsin, Madison, WI 53706, USA}
\author{K. D. de Vries}
\affiliation{Vrije Universiteit Brussel (VUB), Dienst ELEM, B-1050 Brussels, Belgium}
\author{G. de Wasseige}
\affiliation{Vrije Universiteit Brussel (VUB), Dienst ELEM, B-1050 Brussels, Belgium}
\author{M. de With}
\affiliation{Institut f{\"u}r Physik, Humboldt-Universit{\"a}t zu Berlin, D-12489 Berlin, Germany}
\author{T. DeYoung}
\affiliation{Dept. of Physics and Astronomy, Michigan State University, East Lansing, MI 48824, USA}
\author{A. Diaz}
\affiliation{Dept. of Physics, Massachusetts Institute of Technology, Cambridge, MA 02139, USA}
\author{J. C. D{\'\i}az-V{\'e}lez}
\affiliation{Dept. of Physics and Wisconsin IceCube Particle Astrophysics Center, University of Wisconsin, Madison, WI 53706, USA}
\author{H. Dujmovic}
\affiliation{Karlsruhe Institute of Technology, Institut f{\"u}r Kernphysik, D-76021 Karlsruhe, Germany}
\author{M. Dunkman}
\affiliation{Dept. of Physics, Pennsylvania State University, University Park, PA 16802, USA}
\author{E. Dvorak}
\affiliation{Physics Department, South Dakota School of Mines and Technology, Rapid City, SD 57701, USA}
\author{B. Eberhardt}
\affiliation{Dept. of Physics and Wisconsin IceCube Particle Astrophysics Center, University of Wisconsin, Madison, WI 53706, USA}
\author{T. Ehrhardt}
\affiliation{Institute of Physics, University of Mainz, Staudinger Weg 7, D-55099 Mainz, Germany}
\author{P. Eller}
\affiliation{Dept. of Physics, Pennsylvania State University, University Park, PA 16802, USA}
\author{R. Engel}
\affiliation{Karlsruhe Institute of Technology, Institut f{\"u}r Kernphysik, D-76021 Karlsruhe, Germany}
\author{P. A. Evenson}
\affiliation{Bartol Research Institute and Dept. of Physics and Astronomy, University of Delaware, Newark, DE 19716, USA}
\author{S. Fahey}
\affiliation{Dept. of Physics and Wisconsin IceCube Particle Astrophysics Center, University of Wisconsin, Madison, WI 53706, USA}
\author{A. R. Fazely}
\affiliation{Dept. of Physics, Southern University, Baton Rouge, LA 70813, USA}
\author{J. Felde}
\affiliation{Dept. of Physics, University of Maryland, College Park, MD 20742, USA}
\author{K. Filimonov}
\affiliation{Dept. of Physics, University of California, Berkeley, CA 94720, USA}
\author{C. Finley}
\affiliation{Oskar Klein Centre and Dept. of Physics, Stockholm University, SE-10691 Stockholm, Sweden}
\author{D. Fox}
\affiliation{Dept. of Astronomy and Astrophysics, Pennsylvania State University, University Park, PA 16802, USA}
\author{A. Franckowiak}
\affiliation{DESY, D-15738 Zeuthen, Germany}
\author{E. Friedman}
\affiliation{Dept. of Physics, University of Maryland, College Park, MD 20742, USA}
\author{A. Fritz}
\affiliation{Institute of Physics, University of Mainz, Staudinger Weg 7, D-55099 Mainz, Germany}
\author{T. K. Gaisser}
\affiliation{Bartol Research Institute and Dept. of Physics and Astronomy, University of Delaware, Newark, DE 19716, USA}
\author{J. Gallagher}
\affiliation{Dept. of Astronomy, University of Wisconsin, Madison, WI 53706, USA}
\author{E. Ganster}
\affiliation{III. Physikalisches Institut, RWTH Aachen University, D-52056 Aachen, Germany}
\author{S. Garrappa}
\affiliation{DESY, D-15738 Zeuthen, Germany}
\author{L. Gerhardt}
\affiliation{Lawrence Berkeley National Laboratory, Berkeley, CA 94720, USA}
\author{K. Ghorbani}
\affiliation{Dept. of Physics and Wisconsin IceCube Particle Astrophysics Center, University of Wisconsin, Madison, WI 53706, USA}
\author{T. Glauch}
\affiliation{Physik-department, Technische Universit{\"a}t M{\"u}nchen, D-85748 Garching, Germany}
\author{T. Gl{\"u}senkamp}
\affiliation{Erlangen Centre for Astroparticle Physics, Friedrich-Alexander-Universit{\"a}t Erlangen-N{\"u}rnberg, D-91058 Erlangen, Germany}
\author{A. Goldschmidt}
\affiliation{Lawrence Berkeley National Laboratory, Berkeley, CA 94720, USA}
\author{J. G. Gonzalez}
\affiliation{Bartol Research Institute and Dept. of Physics and Astronomy, University of Delaware, Newark, DE 19716, USA}
\author{D. Grant}
\affiliation{Dept. of Physics and Astronomy, Michigan State University, East Lansing, MI 48824, USA}
\author{Z. Griffith}
\affiliation{Dept. of Physics and Wisconsin IceCube Particle Astrophysics Center, University of Wisconsin, Madison, WI 53706, USA}
\author{S. Griswold}
\affiliation{Dept. of Physics and Astronomy, University of Rochester, Rochester, NY 14627, USA}
\author{M. G{\"u}nder}
\affiliation{III. Physikalisches Institut, RWTH Aachen University, D-52056 Aachen, Germany}
\author{M. G{\"u}nd{\"u}z}
\affiliation{Fakult{\"a}t f{\"u}r Physik {\&} Astronomie, Ruhr-Universit{\"a}t Bochum, D-44780 Bochum, Germany}
\author{C. Haack}
\affiliation{III. Physikalisches Institut, RWTH Aachen University, D-52056 Aachen, Germany}
\author{A. Hallgren}
\affiliation{Dept. of Physics and Astronomy, Uppsala University, Box 516, S-75120 Uppsala, Sweden}
\author{R. Halliday}
\affiliation{Dept. of Physics and Astronomy, Michigan State University, East Lansing, MI 48824, USA}
\author{L. Halve}
\affiliation{III. Physikalisches Institut, RWTH Aachen University, D-52056 Aachen, Germany}
\author{F. Halzen}
\affiliation{Dept. of Physics and Wisconsin IceCube Particle Astrophysics Center, University of Wisconsin, Madison, WI 53706, USA}
\author{K. Hanson}
\affiliation{Dept. of Physics and Wisconsin IceCube Particle Astrophysics Center, University of Wisconsin, Madison, WI 53706, USA}
\author{A. Haungs}
\affiliation{Karlsruhe Institute of Technology, Institut f{\"u}r Kernphysik, D-76021 Karlsruhe, Germany}
\author{D. Hebecker}
\affiliation{Institut f{\"u}r Physik, Humboldt-Universit{\"a}t zu Berlin, D-12489 Berlin, Germany}
\author{D. Heereman}
\affiliation{Universit{\'e} Libre de Bruxelles, Science Faculty CP230, B-1050 Brussels, Belgium}
\author{P. Heix}
\affiliation{III. Physikalisches Institut, RWTH Aachen University, D-52056 Aachen, Germany}
\author{K. Helbing}
\affiliation{Dept. of Physics, University of Wuppertal, D-42119 Wuppertal, Germany}
\author{R. Hellauer}
\affiliation{Dept. of Physics, University of Maryland, College Park, MD 20742, USA}
\author{F. Henningsen}
\affiliation{Physik-department, Technische Universit{\"a}t M{\"u}nchen, D-85748 Garching, Germany}
\author{S. Hickford}
\affiliation{Dept. of Physics, University of Wuppertal, D-42119 Wuppertal, Germany}
\author{J. Hignight}
\affiliation{Dept. of Physics, University of Alberta, Edmonton, Alberta, Canada T6G 2E1}
\author{G. C. Hill}
\affiliation{Department of Physics, University of Adelaide, Adelaide, 5005, Australia}
\author{K. D. Hoffman}
\affiliation{Dept. of Physics, University of Maryland, College Park, MD 20742, USA}
\author{R. Hoffmann}
\affiliation{Dept. of Physics, University of Wuppertal, D-42119 Wuppertal, Germany}
\author{T. Hoinka}
\affiliation{Dept. of Physics, TU Dortmund University, D-44221 Dortmund, Germany}
\author{B. Hokanson-Fasig}
\affiliation{Dept. of Physics and Wisconsin IceCube Particle Astrophysics Center, University of Wisconsin, Madison, WI 53706, USA}
\author{K. Hoshina}
\affiliation{Dept. of Physics and Wisconsin IceCube Particle Astrophysics Center, University of Wisconsin, Madison, WI 53706, USA}
\thanks{Earthquake Research Institute, University of Tokyo, Bunkyo, Tokyo 113-0032, Japan}
\author{F. Huang}
\affiliation{Dept. of Physics, Pennsylvania State University, University Park, PA 16802, USA}
\author{M. Huber}
\affiliation{Physik-department, Technische Universit{\"a}t M{\"u}nchen, D-85748 Garching, Germany}
\author{T. Huber}
\affiliation{Karlsruhe Institute of Technology, Institut f{\"u}r Kernphysik, D-76021 Karlsruhe, Germany}
\affiliation{DESY, D-15738 Zeuthen, Germany}
\author{K. Hultqvist}
\affiliation{Oskar Klein Centre and Dept. of Physics, Stockholm University, SE-10691 Stockholm, Sweden}
\author{M. H{\"u}nnefeld}
\affiliation{Dept. of Physics, TU Dortmund University, D-44221 Dortmund, Germany}
\author{R. Hussain}
\affiliation{Dept. of Physics and Wisconsin IceCube Particle Astrophysics Center, University of Wisconsin, Madison, WI 53706, USA}
\author{S. In}
\affiliation{Dept. of Physics, Sungkyunkwan University, Suwon 16419, Korea}
\author{N. Iovine}
\affiliation{Universit{\'e} Libre de Bruxelles, Science Faculty CP230, B-1050 Brussels, Belgium}
\author{A. Ishihara}
\affiliation{Dept. of Physics and Institute for Global Prominent Research, Chiba University, Chiba 263-8522, Japan}
\author{G. S. Japaridze}
\affiliation{CTSPS, Clark-Atlanta University, Atlanta, GA 30314, USA}
\author{M. Jeong}
\affiliation{Dept. of Physics, Sungkyunkwan University, Suwon 16419, Korea}
\author{K. Jero}
\affiliation{Dept. of Physics and Wisconsin IceCube Particle Astrophysics Center, University of Wisconsin, Madison, WI 53706, USA}
\author{B. J. P. Jones}
\affiliation{Dept. of Physics, University of Texas at Arlington, 502 Yates St., Science Hall Rm 108, Box 19059, Arlington, TX 76019, USA}
\author{F. Jonske}
\affiliation{III. Physikalisches Institut, RWTH Aachen University, D-52056 Aachen, Germany}
\author{R. Joppe}
\affiliation{III. Physikalisches Institut, RWTH Aachen University, D-52056 Aachen, Germany}
\author{D. Kang}
\affiliation{Karlsruhe Institute of Technology, Institut f{\"u}r Kernphysik, D-76021 Karlsruhe, Germany}
\author{W. Kang}
\affiliation{Dept. of Physics, Sungkyunkwan University, Suwon 16419, Korea}
\author{A. Kappes}
\affiliation{Institut f{\"u}r Kernphysik, Westf{\"a}lische Wilhelms-Universit{\"a}t M{\"u}nster, D-48149 M{\"u}nster, Germany}
\author{D. Kappesser}
\affiliation{Institute of Physics, University of Mainz, Staudinger Weg 7, D-55099 Mainz, Germany}
\author{T. Karg}
\affiliation{DESY, D-15738 Zeuthen, Germany}
\author{M. Karl}
\affiliation{Physik-department, Technische Universit{\"a}t M{\"u}nchen, D-85748 Garching, Germany}
\author{A. Karle}
\affiliation{Dept. of Physics and Wisconsin IceCube Particle Astrophysics Center, University of Wisconsin, Madison, WI 53706, USA}
\author{U. Katz}
\affiliation{Erlangen Centre for Astroparticle Physics, Friedrich-Alexander-Universit{\"a}t Erlangen-N{\"u}rnberg, D-91058 Erlangen, Germany}
\author{M. Kauer}
\affiliation{Dept. of Physics and Wisconsin IceCube Particle Astrophysics Center, University of Wisconsin, Madison, WI 53706, USA}
\author{J. L. Kelley}
\affiliation{Dept. of Physics and Wisconsin IceCube Particle Astrophysics Center, University of Wisconsin, Madison, WI 53706, USA}
\author{A. Kheirandish}
\affiliation{Dept. of Physics and Wisconsin IceCube Particle Astrophysics Center, University of Wisconsin, Madison, WI 53706, USA}
\author{J. Kim}
\affiliation{Dept. of Physics, Sungkyunkwan University, Suwon 16419, Korea}
\author{T. Kintscher}
\affiliation{DESY, D-15738 Zeuthen, Germany}
\author{J. Kiryluk}
\affiliation{Dept. of Physics and Astronomy, Stony Brook University, Stony Brook, NY 11794-3800, USA}
\author{T. Kittler}
\affiliation{Erlangen Centre for Astroparticle Physics, Friedrich-Alexander-Universit{\"a}t Erlangen-N{\"u}rnberg, D-91058 Erlangen, Germany}
\author{S. R. Klein}
\affiliation{Dept. of Physics, University of California, Berkeley, CA 94720, USA}
\affiliation{Lawrence Berkeley National Laboratory, Berkeley, CA 94720, USA}
\author{R. Koirala}
\affiliation{Bartol Research Institute and Dept. of Physics and Astronomy, University of Delaware, Newark, DE 19716, USA}
\author{H. Kolanoski}
\affiliation{Institut f{\"u}r Physik, Humboldt-Universit{\"a}t zu Berlin, D-12489 Berlin, Germany}
\author{L. K{\"o}pke}
\affiliation{Institute of Physics, University of Mainz, Staudinger Weg 7, D-55099 Mainz, Germany}
\author{C. Kopper}
\affiliation{Dept. of Physics and Astronomy, Michigan State University, East Lansing, MI 48824, USA}
\author{S. Kopper}
\affiliation{Dept. of Physics and Astronomy, University of Alabama, Tuscaloosa, AL 35487, USA}
\author{D. J. Koskinen}
\affiliation{Niels Bohr Institute, University of Copenhagen, DK-2100 Copenhagen, Denmark}
\author{M. Kowalski}
\affiliation{Institut f{\"u}r Physik, Humboldt-Universit{\"a}t zu Berlin, D-12489 Berlin, Germany}
\affiliation{DESY, D-15738 Zeuthen, Germany}
\author{K. Krings}
\affiliation{Physik-department, Technische Universit{\"a}t M{\"u}nchen, D-85748 Garching, Germany}
\author{G. Kr{\"u}ckl}
\affiliation{Institute of Physics, University of Mainz, Staudinger Weg 7, D-55099 Mainz, Germany}
\author{N. Kulacz}
\affiliation{Dept. of Physics, University of Alberta, Edmonton, Alberta, Canada T6G 2E1}
\author{N. Kurahashi}
\affiliation{Dept. of Physics, Drexel University, 3141 Chestnut Street, Philadelphia, PA 19104, USA}
\author{A. Kyriacou}
\affiliation{Department of Physics, University of Adelaide, Adelaide, 5005, Australia}
\author{M. Labare}
\affiliation{Dept. of Physics and Astronomy, University of Gent, B-9000 Gent, Belgium}
\author{J. L. Lanfranchi}
\affiliation{Dept. of Physics, Pennsylvania State University, University Park, PA 16802, USA}
\author{M. J. Larson}
\affiliation{Dept. of Physics, University of Maryland, College Park, MD 20742, USA}
\author{F. Lauber}
\affiliation{Dept. of Physics, University of Wuppertal, D-42119 Wuppertal, Germany}
\author{J. P. Lazar}
\affiliation{Dept. of Physics and Wisconsin IceCube Particle Astrophysics Center, University of Wisconsin, Madison, WI 53706, USA}
\author{K. Leonard}
\affiliation{Dept. of Physics and Wisconsin IceCube Particle Astrophysics Center, University of Wisconsin, Madison, WI 53706, USA}
\author{A. Leszczy{\'n}ska}
\affiliation{Karlsruhe Institute of Technology, Institut f{\"u}r Kernphysik, D-76021 Karlsruhe, Germany}
\author{M. Leuermann}
\affiliation{III. Physikalisches Institut, RWTH Aachen University, D-52056 Aachen, Germany}
\author{Q. R. Liu}
\affiliation{Dept. of Physics and Wisconsin IceCube Particle Astrophysics Center, University of Wisconsin, Madison, WI 53706, USA}
\author{E. Lohfink}
\affiliation{Institute of Physics, University of Mainz, Staudinger Weg 7, D-55099 Mainz, Germany}
\author{C. J. Lozano Mariscal}
\affiliation{Institut f{\"u}r Kernphysik, Westf{\"a}lische Wilhelms-Universit{\"a}t M{\"u}nster, D-48149 M{\"u}nster, Germany}
\author{L. Lu}
\affiliation{Dept. of Physics and Institute for Global Prominent Research, Chiba University, Chiba 263-8522, Japan}
\author{F. Lucarelli}
\affiliation{D{\'e}partement de physique nucl{\'e}aire et corpusculaire, Universit{\'e} de Gen{\`e}ve, CH-1211 Gen{\`e}ve, Switzerland}
\author{J. L{\"u}nemann}
\affiliation{Vrije Universiteit Brussel (VUB), Dienst ELEM, B-1050 Brussels, Belgium}
\author{W. Luszczak}
\affiliation{Dept. of Physics and Wisconsin IceCube Particle Astrophysics Center, University of Wisconsin, Madison, WI 53706, USA}
\author{Y. Lyu}
\affiliation{Dept. of Physics, University of California, Berkeley, CA 94720, USA}
\affiliation{Lawrence Berkeley National Laboratory, Berkeley, CA 94720, USA}
\author{W. Y. Ma}
\affiliation{DESY, D-15738 Zeuthen, Germany}
\author{J. Madsen}
\affiliation{Dept. of Physics, University of Wisconsin, River Falls, WI 54022, USA}
\author{G. Maggi}
\affiliation{Vrije Universiteit Brussel (VUB), Dienst ELEM, B-1050 Brussels, Belgium}
\author{K. B. M. Mahn}
\affiliation{Dept. of Physics and Astronomy, Michigan State University, East Lansing, MI 48824, USA}
\author{Y. Makino}
\affiliation{Dept. of Physics and Institute for Global Prominent Research, Chiba University, Chiba 263-8522, Japan}
\author{P. Mallik}
\affiliation{III. Physikalisches Institut, RWTH Aachen University, D-52056 Aachen, Germany}
\author{K. Mallot}
\affiliation{Dept. of Physics and Wisconsin IceCube Particle Astrophysics Center, University of Wisconsin, Madison, WI 53706, USA}
\author{S. Mancina}
\affiliation{Dept. of Physics and Wisconsin IceCube Particle Astrophysics Center, University of Wisconsin, Madison, WI 53706, USA}
\author{I. C. Mari{\c{s}}}
\affiliation{Universit{\'e} Libre de Bruxelles, Science Faculty CP230, B-1050 Brussels, Belgium}
\author{R. Maruyama}
\affiliation{Dept. of Physics, Yale University, New Haven, CT 06520, USA}
\author{K. Mase}
\affiliation{Dept. of Physics and Institute for Global Prominent Research, Chiba University, Chiba 263-8522, Japan}
\author{R. Maunu}
\affiliation{Dept. of Physics, University of Maryland, College Park, MD 20742, USA}
\author{F. McNally}
\affiliation{Department of Physics, Mercer University, Macon, GA 31207-0001, USA}
\author{K. Meagher}
\affiliation{Dept. of Physics and Wisconsin IceCube Particle Astrophysics Center, University of Wisconsin, Madison, WI 53706, USA}
\author{M. Medici}
\affiliation{Niels Bohr Institute, University of Copenhagen, DK-2100 Copenhagen, Denmark}
\author{A. Medina}
\affiliation{Dept. of Physics and Center for Cosmology and Astro-Particle Physics, Ohio State University, Columbus, OH 43210, USA}
\author{M. Meier}
\affiliation{Dept. of Physics, TU Dortmund University, D-44221 Dortmund, Germany}
\author{S. Meighen-Berger}
\affiliation{Physik-department, Technische Universit{\"a}t M{\"u}nchen, D-85748 Garching, Germany}
\author{T. Menne}
\affiliation{Dept. of Physics, TU Dortmund University, D-44221 Dortmund, Germany}
\author{G. Merino}
\affiliation{Dept. of Physics and Wisconsin IceCube Particle Astrophysics Center, University of Wisconsin, Madison, WI 53706, USA}
\author{T. Meures}
\affiliation{Universit{\'e} Libre de Bruxelles, Science Faculty CP230, B-1050 Brussels, Belgium}
\author{J. Micallef}
\affiliation{Dept. of Physics and Astronomy, Michigan State University, East Lansing, MI 48824, USA}
\author{D. Mockler}
\affiliation{Universit{\'e} Libre de Bruxelles, Science Faculty CP230, B-1050 Brussels, Belgium}
\author{G. Moment{\'e}}
\affiliation{Institute of Physics, University of Mainz, Staudinger Weg 7, D-55099 Mainz, Germany}
\author{T. Montaruli}
\affiliation{D{\'e}partement de physique nucl{\'e}aire et corpusculaire, Universit{\'e} de Gen{\`e}ve, CH-1211 Gen{\`e}ve, Switzerland}
\author{R. W. Moore}
\affiliation{Dept. of Physics, University of Alberta, Edmonton, Alberta, Canada T6G 2E1}
\author{R. Morse}
\affiliation{Dept. of Physics and Wisconsin IceCube Particle Astrophysics Center, University of Wisconsin, Madison, WI 53706, USA}
\author{M. Moulai}
\affiliation{Dept. of Physics, Massachusetts Institute of Technology, Cambridge, MA 02139, USA}
\author{P. Muth}
\affiliation{III. Physikalisches Institut, RWTH Aachen University, D-52056 Aachen, Germany}
\author{R. Nagai}
\affiliation{Dept. of Physics and Institute for Global Prominent Research, Chiba University, Chiba 263-8522, Japan}
\author{U. Naumann}
\affiliation{Dept. of Physics, University of Wuppertal, D-42119 Wuppertal, Germany}
\author{G. Neer}
\affiliation{Dept. of Physics and Astronomy, Michigan State University, East Lansing, MI 48824, USA}
\author{H. Niederhausen}
\affiliation{Physik-department, Technische Universit{\"a}t M{\"u}nchen, D-85748 Garching, Germany}
\author{M. U. Nisa}
\affiliation{Dept. of Physics and Astronomy, Michigan State University, East Lansing, MI 48824, USA}
\author{S. C. Nowicki}
\affiliation{Dept. of Physics and Astronomy, Michigan State University, East Lansing, MI 48824, USA}
\author{D. R. Nygren}
\affiliation{Lawrence Berkeley National Laboratory, Berkeley, CA 94720, USA}
\author{A. Obertacke Pollmann}
\affiliation{Dept. of Physics, University of Wuppertal, D-42119 Wuppertal, Germany}
\author{M. Oehler}
\affiliation{Karlsruhe Institute of Technology, Institut f{\"u}r Kernphysik, D-76021 Karlsruhe, Germany}
\author{A. Olivas}
\affiliation{Dept. of Physics, University of Maryland, College Park, MD 20742, USA}
\author{A. O'Murchadha}
\affiliation{Universit{\'e} Libre de Bruxelles, Science Faculty CP230, B-1050 Brussels, Belgium}
\author{E. O'Sullivan}
\affiliation{Oskar Klein Centre and Dept. of Physics, Stockholm University, SE-10691 Stockholm, Sweden}
\author{T. Palczewski}
\affiliation{Dept. of Physics, University of California, Berkeley, CA 94720, USA}
\affiliation{Lawrence Berkeley National Laboratory, Berkeley, CA 94720, USA}
\author{H. Pandya}
\affiliation{Bartol Research Institute and Dept. of Physics and Astronomy, University of Delaware, Newark, DE 19716, USA}
\author{D. V. Pankova}
\affiliation{Dept. of Physics, Pennsylvania State University, University Park, PA 16802, USA}
\author{N. Park}
\affiliation{Dept. of Physics and Wisconsin IceCube Particle Astrophysics Center, University of Wisconsin, Madison, WI 53706, USA}
\author{P. Peiffer}
\affiliation{Institute of Physics, University of Mainz, Staudinger Weg 7, D-55099 Mainz, Germany}
\author{C. P{\'e}rez de los Heros}
\affiliation{Dept. of Physics and Astronomy, Uppsala University, Box 516, S-75120 Uppsala, Sweden}
\author{S. Philippen}
\affiliation{III. Physikalisches Institut, RWTH Aachen University, D-52056 Aachen, Germany}
\author{D. Pieloth}
\affiliation{Dept. of Physics, TU Dortmund University, D-44221 Dortmund, Germany}
\author{E. Pinat}
\affiliation{Universit{\'e} Libre de Bruxelles, Science Faculty CP230, B-1050 Brussels, Belgium}
\author{A. Pizzuto}
\affiliation{Dept. of Physics and Wisconsin IceCube Particle Astrophysics Center, University of Wisconsin, Madison, WI 53706, USA}
\author{M. Plum}
\affiliation{Department of Physics, Marquette University, Milwaukee, WI, 53201, USA}
\author{A. Porcelli}
\affiliation{Dept. of Physics and Astronomy, University of Gent, B-9000 Gent, Belgium}
\author{P. B. Price}
\affiliation{Dept. of Physics, University of California, Berkeley, CA 94720, USA}
\author{G. T. Przybylski}
\affiliation{Lawrence Berkeley National Laboratory, Berkeley, CA 94720, USA}
\author{C. Raab}
\affiliation{Universit{\'e} Libre de Bruxelles, Science Faculty CP230, B-1050 Brussels, Belgium}
\author{A. Raissi}
\affiliation{Dept. of Physics and Astronomy, University of Canterbury, Private Bag 4800, Christchurch, New Zealand}
\author{M. Rameez}
\affiliation{Niels Bohr Institute, University of Copenhagen, DK-2100 Copenhagen, Denmark}
\author{L. Rauch}
\affiliation{DESY, D-15738 Zeuthen, Germany}
\author{K. Rawlins}
\affiliation{Dept. of Physics and Astronomy, University of Alaska Anchorage, 3211 Providence Dr., Anchorage, AK 99508, USA}
\author{I. C. Rea}
\affiliation{Physik-department, Technische Universit{\"a}t M{\"u}nchen, D-85748 Garching, Germany}
\author{R. Reimann}
\affiliation{III. Physikalisches Institut, RWTH Aachen University, D-52056 Aachen, Germany}
\author{B. Relethford}
\affiliation{Dept. of Physics, Drexel University, 3141 Chestnut Street, Philadelphia, PA 19104, USA}
\author{M. Renschler}
\affiliation{Karlsruhe Institute of Technology, Institut f{\"u}r Kernphysik, D-76021 Karlsruhe, Germany}
\author{G. Renzi}
\affiliation{Universit{\'e} Libre de Bruxelles, Science Faculty CP230, B-1050 Brussels, Belgium}
\author{E. Resconi}
\affiliation{Physik-department, Technische Universit{\"a}t M{\"u}nchen, D-85748 Garching, Germany}
\author{W. Rhode}
\affiliation{Dept. of Physics, TU Dortmund University, D-44221 Dortmund, Germany}
\author{M. Richman}
\affiliation{Dept. of Physics, Drexel University, 3141 Chestnut Street, Philadelphia, PA 19104, USA}
\author{S. Robertson}
\affiliation{Lawrence Berkeley National Laboratory, Berkeley, CA 94720, USA}
\author{M. Rongen}
\affiliation{III. Physikalisches Institut, RWTH Aachen University, D-52056 Aachen, Germany}
\author{C. Rott}
\affiliation{Dept. of Physics, Sungkyunkwan University, Suwon 16419, Korea}
\author{T. Ruhe}
\affiliation{Dept. of Physics, TU Dortmund University, D-44221 Dortmund, Germany}
\author{D. Ryckbosch}
\affiliation{Dept. of Physics and Astronomy, University of Gent, B-9000 Gent, Belgium}
\author{D. Rysewyk}
\affiliation{Dept. of Physics and Astronomy, Michigan State University, East Lansing, MI 48824, USA}
\author{I. Safa}
\affiliation{Dept. of Physics and Wisconsin IceCube Particle Astrophysics Center, University of Wisconsin, Madison, WI 53706, USA}
\author{S. E. Sanchez Herrera}
\affiliation{Dept. of Physics and Astronomy, Michigan State University, East Lansing, MI 48824, USA}
\author{A. Sandrock}
\affiliation{Dept. of Physics, TU Dortmund University, D-44221 Dortmund, Germany}
\author{J. Sandroos}
\affiliation{Institute of Physics, University of Mainz, Staudinger Weg 7, D-55099 Mainz, Germany}
\author{M. Santander}
\affiliation{Dept. of Physics and Astronomy, University of Alabama, Tuscaloosa, AL 35487, USA}
\author{S. Sarkar}
\affiliation{Dept. of Physics, University of Oxford, Parks Road, Oxford OX1 3PU, UK}
\author{S. Sarkar}
\affiliation{Dept. of Physics, University of Alberta, Edmonton, Alberta, Canada T6G 2E1}
\author{K. Satalecka}
\affiliation{DESY, D-15738 Zeuthen, Germany}
\author{M. Schaufel}
\affiliation{III. Physikalisches Institut, RWTH Aachen University, D-52056 Aachen, Germany}
\author{H. Schieler}
\affiliation{Karlsruhe Institute of Technology, Institut f{\"u}r Kernphysik, D-76021 Karlsruhe, Germany}
\author{P. Schlunder}
\affiliation{Dept. of Physics, TU Dortmund University, D-44221 Dortmund, Germany}
\author{T. Schmidt}
\affiliation{Dept. of Physics, University of Maryland, College Park, MD 20742, USA}
\author{A. Schneider}
\affiliation{Dept. of Physics and Wisconsin IceCube Particle Astrophysics Center, University of Wisconsin, Madison, WI 53706, USA}
\author{J. Schneider}
\affiliation{Erlangen Centre for Astroparticle Physics, Friedrich-Alexander-Universit{\"a}t Erlangen-N{\"u}rnberg, D-91058 Erlangen, Germany}
\author{F. G. Schr{\"o}der}
\affiliation{Karlsruhe Institute of Technology, Institut f{\"u}r Kernphysik, D-76021 Karlsruhe, Germany}
\affiliation{Bartol Research Institute and Dept. of Physics and Astronomy, University of Delaware, Newark, DE 19716, USA}
\author{L. Schumacher}
\affiliation{III. Physikalisches Institut, RWTH Aachen University, D-52056 Aachen, Germany}
\author{S. Sclafani}
\affiliation{Dept. of Physics, Drexel University, 3141 Chestnut Street, Philadelphia, PA 19104, USA}
\author{D. Seckel}
\affiliation{Bartol Research Institute and Dept. of Physics and Astronomy, University of Delaware, Newark, DE 19716, USA}
\author{S. Seunarine}
\affiliation{Dept. of Physics, University of Wisconsin, River Falls, WI 54022, USA}
\author{S. Shefali}
\affiliation{III. Physikalisches Institut, RWTH Aachen University, D-52056 Aachen, Germany}
\author{M. Silva}
\affiliation{Dept. of Physics and Wisconsin IceCube Particle Astrophysics Center, University of Wisconsin, Madison, WI 53706, USA}
\author{R. Snihur}
\affiliation{Dept. of Physics and Wisconsin IceCube Particle Astrophysics Center, University of Wisconsin, Madison, WI 53706, USA}
\author{J. Soedingrekso}
\affiliation{Dept. of Physics, TU Dortmund University, D-44221 Dortmund, Germany}
\author{D. Soldin}
\affiliation{Bartol Research Institute and Dept. of Physics and Astronomy, University of Delaware, Newark, DE 19716, USA}
\author{M. Song}
\affiliation{Dept. of Physics, University of Maryland, College Park, MD 20742, USA}
\author{G. M. Spiczak}
\affiliation{Dept. of Physics, University of Wisconsin, River Falls, WI 54022, USA}
\author{C. Spiering}
\affiliation{DESY, D-15738 Zeuthen, Germany}
\author{J. Stachurska}
\affiliation{DESY, D-15738 Zeuthen, Germany}
\author{M. Stamatikos}
\affiliation{Dept. of Physics and Center for Cosmology and Astro-Particle Physics, Ohio State University, Columbus, OH 43210, USA}
\author{T. Stanev}
\affiliation{Bartol Research Institute and Dept. of Physics and Astronomy, University of Delaware, Newark, DE 19716, USA}
\author{R. Stein}
\affiliation{DESY, D-15738 Zeuthen, Germany}
\author{P. Steinm{\"u}ller}
\affiliation{Karlsruhe Institute of Technology, Institut f{\"u}r Kernphysik, D-76021 Karlsruhe, Germany}
\author{J. Stettner}
\affiliation{III. Physikalisches Institut, RWTH Aachen University, D-52056 Aachen, Germany}
\author{A. Steuer}
\affiliation{Institute of Physics, University of Mainz, Staudinger Weg 7, D-55099 Mainz, Germany}
\author{T. Stezelberger}
\affiliation{Lawrence Berkeley National Laboratory, Berkeley, CA 94720, USA}
\author{R. G. Stokstad}
\affiliation{Lawrence Berkeley National Laboratory, Berkeley, CA 94720, USA}
\author{A. St{\"o}{\ss}l}
\affiliation{Dept. of Physics and Institute for Global Prominent Research, Chiba University, Chiba 263-8522, Japan}
\author{N. L. Strotjohann}
\affiliation{DESY, D-15738 Zeuthen, Germany}
\author{T. St{\"u}rwald}
\affiliation{III. Physikalisches Institut, RWTH Aachen University, D-52056 Aachen, Germany}
\author{T. Stuttard}
\affiliation{Niels Bohr Institute, University of Copenhagen, DK-2100 Copenhagen, Denmark}
\author{G. W. Sullivan}
\affiliation{Dept. of Physics, University of Maryland, College Park, MD 20742, USA}
\author{I. Taboada}
\affiliation{School of Physics and Center for Relativistic Astrophysics, Georgia Institute of Technology, Atlanta, GA 30332, USA}
\author{F. Tenholt}
\affiliation{Fakult{\"a}t f{\"u}r Physik {\&} Astronomie, Ruhr-Universit{\"a}t Bochum, D-44780 Bochum, Germany}
\author{S. Ter-Antonyan}
\affiliation{Dept. of Physics, Southern University, Baton Rouge, LA 70813, USA}
\author{A. Terliuk}
\affiliation{DESY, D-15738 Zeuthen, Germany}
\author{S. Tilav}
\affiliation{Bartol Research Institute and Dept. of Physics and Astronomy, University of Delaware, Newark, DE 19716, USA}
\author{K. Tollefson}
\affiliation{Dept. of Physics and Astronomy, Michigan State University, East Lansing, MI 48824, USA}
\author{L. Tomankova}
\affiliation{Fakult{\"a}t f{\"u}r Physik {\&} Astronomie, Ruhr-Universit{\"a}t Bochum, D-44780 Bochum, Germany}
\author{C. T{\"o}nnis}
\affiliation{Institute of Basic Science, Sungkyunkwan University, Suwon 16419, Korea}
\author{S. Toscano}
\affiliation{Universit{\'e} Libre de Bruxelles, Science Faculty CP230, B-1050 Brussels, Belgium}
\author{D. Tosi}
\affiliation{Dept. of Physics and Wisconsin IceCube Particle Astrophysics Center, University of Wisconsin, Madison, WI 53706, USA}
\author{A. Trettin}
\affiliation{DESY, D-15738 Zeuthen, Germany}
\author{M. Tselengidou}
\affiliation{Erlangen Centre for Astroparticle Physics, Friedrich-Alexander-Universit{\"a}t Erlangen-N{\"u}rnberg, D-91058 Erlangen, Germany}
\author{C. F. Tung}
\affiliation{School of Physics and Center for Relativistic Astrophysics, Georgia Institute of Technology, Atlanta, GA 30332, USA}
\author{A. Turcati}
\affiliation{Physik-department, Technische Universit{\"a}t M{\"u}nchen, D-85748 Garching, Germany}
\author{R. Turcotte}
\affiliation{Karlsruhe Institute of Technology, Institut f{\"u}r Kernphysik, D-76021 Karlsruhe, Germany}
\author{C. F. Turley}
\affiliation{Dept. of Physics, Pennsylvania State University, University Park, PA 16802, USA}
\author{B. Ty}
\affiliation{Dept. of Physics and Wisconsin IceCube Particle Astrophysics Center, University of Wisconsin, Madison, WI 53706, USA}
\author{E. Unger}
\affiliation{Dept. of Physics and Astronomy, Uppsala University, Box 516, S-75120 Uppsala, Sweden}
\author{M. A. Unland Elorrieta}
\affiliation{Institut f{\"u}r Kernphysik, Westf{\"a}lische Wilhelms-Universit{\"a}t M{\"u}nster, D-48149 M{\"u}nster, Germany}
\author{M. Usner}
\affiliation{DESY, D-15738 Zeuthen, Germany}
\author{J. Vandenbroucke}
\affiliation{Dept. of Physics and Wisconsin IceCube Particle Astrophysics Center, University of Wisconsin, Madison, WI 53706, USA}
\author{W. Van Driessche}
\affiliation{Dept. of Physics and Astronomy, University of Gent, B-9000 Gent, Belgium}
\author{D. van Eijk}
\affiliation{Dept. of Physics and Wisconsin IceCube Particle Astrophysics Center, University of Wisconsin, Madison, WI 53706, USA}
\author{N. van Eijndhoven}
\affiliation{Vrije Universiteit Brussel (VUB), Dienst ELEM, B-1050 Brussels, Belgium}
\author{S. Vanheule}
\affiliation{Dept. of Physics and Astronomy, University of Gent, B-9000 Gent, Belgium}
\author{J. van Santen}
\affiliation{DESY, D-15738 Zeuthen, Germany}
\author{M. Vraeghe}
\affiliation{Dept. of Physics and Astronomy, University of Gent, B-9000 Gent, Belgium}
\author{C. Walck}
\affiliation{Oskar Klein Centre and Dept. of Physics, Stockholm University, SE-10691 Stockholm, Sweden}
\author{A. Wallace}
\affiliation{Department of Physics, University of Adelaide, Adelaide, 5005, Australia}
\author{M. Wallraff}
\affiliation{III. Physikalisches Institut, RWTH Aachen University, D-52056 Aachen, Germany}
\author{N. Wandkowsky}
\affiliation{Dept. of Physics and Wisconsin IceCube Particle Astrophysics Center, University of Wisconsin, Madison, WI 53706, USA}
\author{T. B. Watson}
\affiliation{Dept. of Physics, University of Texas at Arlington, 502 Yates St., Science Hall Rm 108, Box 19059, Arlington, TX 76019, USA}
\author{C. Weaver}
\affiliation{Dept. of Physics, University of Alberta, Edmonton, Alberta, Canada T6G 2E1}
\author{A. Weindl}
\affiliation{Karlsruhe Institute of Technology, Institut f{\"u}r Kernphysik, D-76021 Karlsruhe, Germany}
\author{M. J. Weiss}
\affiliation{Dept. of Physics, Pennsylvania State University, University Park, PA 16802, USA}
\author{J. Weldert}
\affiliation{Institute of Physics, University of Mainz, Staudinger Weg 7, D-55099 Mainz, Germany}
\author{C. Wendt}
\affiliation{Dept. of Physics and Wisconsin IceCube Particle Astrophysics Center, University of Wisconsin, Madison, WI 53706, USA}
\author{J. Werthebach}
\affiliation{Dept. of Physics and Wisconsin IceCube Particle Astrophysics Center, University of Wisconsin, Madison, WI 53706, USA}
\author{B. J. Whelan}
\affiliation{Department of Physics, University of Adelaide, Adelaide, 5005, Australia}
\author{N. Whitehorn}
\affiliation{Department of Physics and Astronomy, UCLA, Los Angeles, CA 90095, USA}
\author{K. Wiebe}
\affiliation{Institute of Physics, University of Mainz, Staudinger Weg 7, D-55099 Mainz, Germany}
\author{C. H. Wiebusch}
\affiliation{III. Physikalisches Institut, RWTH Aachen University, D-52056 Aachen, Germany}
\author{L. Wille}
\affiliation{Dept. of Physics and Wisconsin IceCube Particle Astrophysics Center, University of Wisconsin, Madison, WI 53706, USA}
\author{D. R. Williams}
\affiliation{Dept. of Physics and Astronomy, University of Alabama, Tuscaloosa, AL 35487, USA}
\author{L. Wills}
\affiliation{Dept. of Physics, Drexel University, 3141 Chestnut Street, Philadelphia, PA 19104, USA}
\author{M. Wolf}
\affiliation{Physik-department, Technische Universit{\"a}t M{\"u}nchen, D-85748 Garching, Germany}
\author{J. Wood}
\affiliation{Dept. of Physics and Wisconsin IceCube Particle Astrophysics Center, University of Wisconsin, Madison, WI 53706, USA}
\author{T. R. Wood}
\affiliation{Dept. of Physics, University of Alberta, Edmonton, Alberta, Canada T6G 2E1}
\author{K. Woschnagg}
\affiliation{Dept. of Physics, University of California, Berkeley, CA 94720, USA}
\author{G. Wrede}
\affiliation{Erlangen Centre for Astroparticle Physics, Friedrich-Alexander-Universit{\"a}t Erlangen-N{\"u}rnberg, D-91058 Erlangen, Germany}
\author{D. L. Xu}
\affiliation{Dept. of Physics and Wisconsin IceCube Particle Astrophysics Center, University of Wisconsin, Madison, WI 53706, USA}
\author{X. W. Xu}
\affiliation{Dept. of Physics, Southern University, Baton Rouge, LA 70813, USA}
\author{Y. Xu}
\affiliation{Dept. of Physics and Astronomy, Stony Brook University, Stony Brook, NY 11794-3800, USA}
\author{J. P. Yanez}
\affiliation{Dept. of Physics, University of Alberta, Edmonton, Alberta, Canada T6G 2E1}
\author{G. Yodh}
\affiliation{Dept. of Physics and Astronomy, University of California, Irvine, CA 92697, USA}
\author{S. Yoshida}
\affiliation{Dept. of Physics and Institute for Global Prominent Research, Chiba University, Chiba 263-8522, Japan}
\author{T. Yuan}
\affiliation{Dept. of Physics and Wisconsin IceCube Particle Astrophysics Center, University of Wisconsin, Madison, WI 53706, USA}
\author{M. Z{\"o}cklein}
\affiliation{III. Physikalisches Institut, RWTH Aachen University, D-52056 Aachen, Germany}
\date{\today}

\pacs{}

\begin{abstract}
IceCube has performed several all-sky searches for point-like neutrino sources. This paper accompanies the public data release of track-like neutrino candidates detected by IceCube between April 6, 2008 and July 10, 2018. The selection includes through-going tracks, {\it i.e.}, muon neutrino candidates that reach the detector from all directions, as well as neutrino track events that start within the instrumented volume. This data release encompasses events included in earlier data releases with improved event selection and reconstruction.
\end{abstract}

\maketitle

\section{The Icecube Neutrino Observatory}

Interactions of neutrinos in IceCube are detected using Cerenkov light
emitted by relativistic charged secondary particles. To serve this purpose, one
cubic kilometer of Antarctic ice was instrumented at the South
Pole~\citep{Achterberg:2006md}. A total of 5160 digital optical modules (DOMs)
detect light emission in the ice at a depth ranging from 1450 to $2450\:$m. The
DOMs consist of a photomultiplier tube, electronics for digitization, and LEDs
for detector calibration \citep{Abbasi:2008aa, Abbasi:2010vc}.  DOMs are
attached to 86 cables (called \emph{strings}) in groups of 60 with vertical
spacing of $17\:$m; the mean distance between neighboring strings is
$\sim125\:\mathrm{m}$.  The eight innermost strings form a denser sub-array
called \emph{DeepCore} which targets lower energies~\citep{Abbasi:2011ym}. The
South Pole IceCube Neutrino Observatory includes the surface array called
\emph{IceTop}~\citep{Abbasi:2012nn} that detects and reconstructs air showers
above $300\:$TeV using 82 ice tanks. In the analysis presented here, IceTop's
capabilities are used to veto cosmic-ray-induced backgrounds.

Three event topologies are taken
into account when considering neutrino interactions in IceCube. \emph{Tracks}
are induced by muons traversing the detector. Below $700\:$GeV, muons lose
energy mainly due to ionization; above $700\:$GeV, stochastic energy losses due
to radiative emission become the dominant component. At TeV energies, muons
travel long distances, larger than several kilometers in the Antarctic
ice~\citep{Chirkin:2004hz}. Light is constantly emitted along the track. The
resulting long lever arm gives a good reconstruction performance with median
angular resolution $\Delta\Psi<1\deg$i.\footnote{Evaluated using dedicated Monte
Carlo simulation, verified using experimental data of cosmic-ray shadowing of
the Moon~\citep{Aartsen:2013zka}.} Moreover, the event rate greatly increases
because neutrinos can interact far outside the detector prior to the detection
of the secondary muon with IceCube.  Charged-current interactions of electron
or tau neutrinos, as well as neutral current interactions of any neutrino type,
produce \emph{shower}- or \emph{cascade}-like events. These types of
interactions produce almost spherically symmetric light emission, giving a
median angular resolution of $\sim$10$\deg$. Another topology is induced by
very high energy charged-current $\nu_\tau+\bar{\nu}_\tau$ interactions with
the tau lepton decaying to hadrons after traveling a distance
$\sim50\:\mathrm{m\,PeV}^{-1}\times E_\tau$, resulting in two cascades separated
distinctly. Such a \emph{double-bang} has not been observed so
far~\citep{Aartsen:2015dlt}.

Track-like events are more suited than cascades to search for very localized
(point-like), faint sources using high statistics and good angular resolution.
Such events originate primarily in neutrino charged-current interactions
of muon (anti)neutrinos with nucleons, but also in similar interactions of tau
neutrinos with the tau lepton decaying to muons and neutrinos, or
interactions of electron antineutrinos with electrons by resonant s--channel
$W^-$ exchange~\citep[Glashow Resonance;][]{Glashow:1960zz}. The energy of
a muon track is restricted to the fraction of the track visible in the
detector, limiting the energy estimation to the \emph{deposited energy} in
the detector or the muon energy upon entering the detector. Independent of the
details of neutrino production, the incident flux of astrophysical neutrinos at
Earth will consist of an approximately equally shared flavor ratio for all
neutrinos due to very long baseline neutrino oscillations~\citep{Athar:2005wg}.
In the following, only the component of tracks created in muon neutrino
interactions is taken into account as the signal of astrophysical neutrinos,
likewise to \cite{Aartsen:2014cva} and \cite{Adrian-Martinez:2014wzf}. The
impact of track events originating from other neutrino interactions is
discussed in \ref{sec:sys}.

Construction of IceCube finished in December~2010 after 6~yr of
deployment. During construction, partial configurations of the detector were
successfully taking data, commonly indicated by IC$XY$, with $XY$ denoting the
number of active strings. The first three years of the event sample are in
partial detector configurations IC40, IC59, and IC79. The previously published
analysis~\citep{Aartsen:2014cva} included the first year of data-taking with
the completed detector IC86 and 3~yr in partial
configuration~\citep{Abbasi:2010rd, Aartsen:2013uuv}. These samples focus on
through-going track-like events, yielding high statistics over broad energy
ranges. In the southern sky, an additional selection of starting tracks, which
has a greatly reduced background rate, is performed access lower
energies~\citep{Aartsen:2016tpb}.  This uses completely independent events to
the ones selected in the aforementioned through-going track channel.

%\begin{deluxetable}{p{0.08\textwidth}rrrr}
%\tablecaption{IceCube samples used in this analysis. For each sample,
%              characteristic features are quoted, separated in the two halves
%              of the sky. Previously published results used four years of data:
%              IC40~\citep{Abbasi:2010rd}, IC59+IC79~\citep{Aartsen:2013uuv},
%              and the first year of IC86~\citep{Aartsen:2014cva}. Data taken in
%              the seasons from 2012 to 2015 are added in this paper. The
%              separation in the two regions is done at a declination of
%              $\delta=-5\deg$. In addition, starting tracks are used in a
%              separate sample~\citep{Aartsen:2016tpb} with two additional years
%              of available data.\label{tab:sample-stats}}
%
%\tablehead{\colhead{Sample} & \colhead{Livetime} & \colhead{atm. $\nu$}
%            & \colhead{Up-going} & \colhead{Down-going}\\
%           & \colhead{(days)} & \colhead{(day$^{-1}$)}
%            & \colhead{events} & \colhead{events}}
%\startdata
%IC40          & 376   &  40         &  16323 &  20577 \\*
%IC59          & 348   & 120         &  48105 &  58906 \\*
%IC79          & 316   & 180         &  54823 &  38310 \\*
%IC86          & 333   & 210         &  67938 &  68302 \\*
%2012-2015     & 1058  & 220         & 235602 & 102983 \\*
%$\Sigma$      & 2431 & \nodata      & 422791 & 289078 \\*
%starting tracks & 1715  & $<\,0.03$ &      0 &    961 \\*
%\enddata
%\end{deluxetable}

The details of all samples are listed in \ref{tab:sample-stats}, including
the exposure time and sample size. In this work, the detector livetime is
increased by adding data from June~2012 to June~2015 to the analysis, thus
increasing the livetime by 1058~days, to a total of 2431~days. The sample of
starting tracks~\citep{Aartsen:2016tpb} has an increased livetime of 2~yr to
a total of 1715~days that coincide with the five most recent years of the
through-going track event sample, starting with IC79.

In the following, the modeling of signal and background for both the
through-going and starting track samples is described.  The event selection is
briefly discussed and the performance of the event sample highlighted. For more
detailed information, refer to \cite{Abbasi:2010rd}, \cite{Aartsen:2013uuv},
and \cite{Aartsen:2014cva} for through-going muons and \cite{Aartsen:2016tpb}
for starting tracks.

\section{Data Selection}

\subsection{Through-Going Tracks}

The main background regarding neutrino searches in IceCube consists of
atmospheric muons that are created in extensive air showers and reach IceCube
at a depth of $\sim2\:\mathrm{km}$.  Almost all events triggering the detector
at a rate of $\sim2.8\:\mathrm{kHz}$ belong to this component. Similar
to~\cite{Aartsen:2014cva}, the selection is split into two regions divided at
the horizon (declination $\delta=-5\deg$) due to different background
characteristics, as explained in the following.

In the northern sky (up-going region, $\delta\geq-5\deg$), the main background
consists of atmospheric muon events that are mis-reconstructed as up-going;
truly up-going muons can only originate from prior neutrino interactions, since
all other muons are shielded by the Earth. Hence, the atmospheric muon
background is rejected by identifying poorly reconstructed events. In order to
achieve this, multivariate selection techniques (boosted decision tree, BDT)
are used to discriminate well-reconstructed tracks from neutrino interactions
against mis-reconstructed background. The variables used in the BDT are
connected to the event quality and a clear track-like topology. Similar to
\cite{Aartsen:2014cva}, BDTs are trained for two signal energy spectra of
$E^{-2}$ and $E^{-2.7}$ and the cut on the linear combination of BDT scores is
optimized to yield the best sensitivity and discovery potential over a wide
range of energies with the final cut on BDT output. These spectra are chosen to
be sensitive to both hard energy spectra as well as soft or cutoff spectra. A
neutrino-dominated sample is obtained in the northern sky. The remaining
background consists of atmospheric neutrinos produced in the northern sky.
These neutrinos are an irreducible background, but follow a softer energy
spectrum ($\sim E^{-3.7}$--$E^{-4.0}$ or $\sim E^{-2.7}$--$E^{-3.0}$ for
conventional or prompt neutrinos, respectively) than the expected signal ($\sim
E^{-2}$ consistent with diffuse muon neutrino signal at $250\:\mathrm{TeV}$ and
above; see \cite{Aartsen:2016xlq}). \ref{fig:effA} shows the effective area
calculated using muon neutrino Monte Carlo simulation of the final event sample
for different declination regions.  In the northern sky (red, blue), a low
energy threshold is achieved, while for near-vertically up-going events
absorption becomes an important effect above $100\:\mathrm{TeV}$.
\ref{fig:mrs} shows the median angular resolution of the track
reconstruction (solid) with respect to the primary neutrino direction against
neutrino energy. Above TeV energies, the kinematic angle (dotted) of muon and
parent neutrino becomes negligible and the angular resolution is below $1\deg$.
The energy resolution of the track's energy proxy is $\sim30\%$ in
$\log_{10}{E}$~\citep{Aartsen:2013vja}. This is only a lower limit on the
energy of a muon that enters the detector from outside and loses energy prior
to detection, as well as on energy of the primary neutrino that produced the
muon in an interaction with a nucleus. For more information about the neutrino
and muon energy estimation, refer to \cite{Aartsen:2015rwa, Aartsen:2016xlq}.

In the southern sky (down-going, $\delta<-5\deg$), the picture changes because
of large backgrounds of well-reconstructed down-going atmospheric muons.
Moreover, muons are produced at high multiplicity in cosmic-ray showers,
resulting in bundles of muons; such bundles produce large amounts of light in
the detector, thus imitating the signature of a single muon of much higher
energy. Similar to the northern sky, BDTs are used to select only the
best-reconstructed events at the highest energies. Following the development of
\cite{Aartsen:2014cva}, in addition to event quality and track topology
parameters, four variables are used to further discriminate atmospheric
muon bundles from single muons. These variables use the deposited energy along
the track, as well as the light-arrival time of photons at the DOMs. The
Cerenkov light yield for high-energy muons is dominated by stochastic cascades
from energy losses along the track superimposed to the Cerenkov cone of the
muon track. Bundles of muons at lower energies show less frequent losses by
stochastic cascades. The result is a smoother light yield along the
track. Furthermore, muon bundles consist of a superposition of many
Cherenkov-cones, resulting in many photons arriving earlier than under the
assumption of one single Cherenkov cone. For a signal-spectrum $E^{-2}$, a BDT
is trained for final selection of events. The large backgrounds require harsh
cuts to reduce their rate significantly, resulting in an effective selection of
only very high energy events, as shown in \ref{fig:effA} (yellow and
purple).  It is evident that the energy threshold in the down-going region
increases to $\sim100\:\mathrm{TeV}$ and even further for more vertically
down-going events. The IceTop surface array is used as an active veto against
coincident air-shower events for vertically down-going events.  For
high energies, this vetoes $90\%$ of the events for vertically down-going
events and less for inclined events, with random coincidences in less than
$0.1\%$ of the cases \citep{Aartsen:2013uuv}; compare \ref{fig:cosSpline}.
The final event rate in the southern sky is optimized to yield the best
sensitivity and discovery potential for an $E^{-2}$ spectrum.

The distribution of the cosine zenith $\cos\theta$ (equivalent to the
negative sine of declination $-\sin\delta$ in equatorial coordinates) is shown
in \ref{fig:cosSpline}. The event rate for experimental data of all 3~yr is
compared to the expectations of muon neutrinos and atmospheric mouns estimated
from Monte Carlo simulation. In the northern sky ($\theta>85\deg$), the sample
is dominated by atmospheric neutrinos produced in the decays of kaons and pions
in cosmic-ray air showers~\citep{Honda:2006qj} and is well described by Monte
Carlo simulation. In the southern sky, the neutrino event rate reduces
drastically due to the higher energy threshold mentioned before.  Instead,
atmospheric muons are the dominant component. Astrophysical neutrinos with hard
energy spectra (as, e.g. shown with $E^{-2}$) do not suffer a rate loss as
severe as for the soft energy spectrum of atmospheric neutrinos. For very
vertically down-going events, the IceTop surface array vetoes atmospheric muon
and neutrino events because of their coincident air shower. In this region, the
observed event rate is kept constant with the rest of the southern sky using
looser cuts on the BDT score, which allows more neutrinos to be detected
especially at the lower energy end. Hence, the corresponding event rate in
\ref{fig:cosSpline} does not decrease above $\cos\theta>0.7$.

\subsection{Starting Tracks}

In the southern sky, the large background of atmospheric muons reduces the
efficiency to select through-going tracks induced by neutrinos below the PeV
regime. A very large fraction of the aforementioned background can be rejected
by imposing an active veto at the detector boundary, as, for example, used in
\cite{Aartsen:2013jdh}. This reduces the detector volume to a smaller fraction
of the instrumented volume sacrificing statistics for signal purity.
Furthermore, the more clearly an event is identified as a starting track, the
more probable it is to be an astrophysical rather than atmospheric background.
Down-going atmospheric neutrino events at high energy are likely to be
accompanied by muons produced in the same cosmic-ray shower that triggers the
veto and reduces the atmospheric neutrino background~\citep{Schonert:2008is,
Gaisser:2014bja}. In analyses using veto techniques~\citep{Aartsen:2013jdh,
Aartsen:2014muf}, the selection is usually more efficient for cascade-like
events than tracks, and high astrophysical neutrino purity demands neglecting
energies below $60\:$TeV where backgrounds are more abundant.  In searches for
point-like sources of astrophysical neutrinos, track-like events are of great
importance given their good angular resolution compared to cascade-like events.
Furthermore, the purity demands are lower since the signal of a point-like
source is restricted to a small portion of the sky, hence reducing the
background significantly. Consequently, the minimum required total charge
deposited in the PMTs of the IceCube detector by an event is lowered to
$1500\:\mathrm{p.e.}$ compared to
$6000\:\mathrm{p.e.}$~\citep{Aartsen:2013jdh}, resulting in a higher signal
efficiency at lower energies. In addition, only down-going tracks are used, and
cuts are imposed that select well-reconstructed track-like
events~\citep{Aartsen:2016tpb}. For $\nu_\mu+\bar{\nu}_\mu$ events at energies
smaller than $200\:\mathrm{TeV}$, the effective area of the analysis is bigger
than for vertically through-going tracks
($\delta<-30\deg$; \ref{fig:effA}).  For energies up to
$1\:$PeV, the effective area is smaller, but a higher purity is achieved.  The
angular resolution for starting tracks is shown in \ref{fig:mrs} (dashed)
and is $\sim1\deg$ in the interesting energy region; the reconstruction is
worse than for through-going events (solid), due to a smaller lever arm for
tracks starting within the fiducial volume of the detector.

In 5~yr, 961 events were recorded in the southern sky starting track
sample. The overlap of events in the starting track sample and the
through-going events is very small, and overlapping events are removed from the
through-going sample because of its higher background rate.

\section{Simulated Detector Response}

\subsection{Effective Area}

In this work, muon tracks induced by astrophysical neutrino interactions are
the main signal category in the search for point-like sources of neutrinos.
Detailed Monte Carlo simulation is used to evaluate the response of IceCube to
such events and distinguish them from atmospheric backgrounds. The median
angular resolution $\Delta\Psi$ and the event rate expectation
\begin{align}
    \dot{N}_\nu=\int d\Omega\int_0^\infty dE\,A_\mathrm{eff}\left(E, \Omega\right)
                \times F_\nu\left(E_\nu,\Omega\right)
    \label{eq:ev_rate}
\end{align}
given by the detector \emph{effective
area} $A_\mathrm{eff}$ and the incident neutrino flux $F_\nu$ can be derived
from the simulation (for more information about the details of the Monte Carlo
simulation used here, see \cite{Aartsen:2016xlq}).

\subsection{Angular Reconstruction}

\subsection{Energy Reconstruction}

\section{Comparison to Previous Releases}

{\ldots here would go the TXS comparison \ldots}

\begin{acknowledgements}
\input{agencies.tex}
\end{acknowledgements}

\bibliographystyle{utphys_mod}

\bibliography{references}

\end{document}

